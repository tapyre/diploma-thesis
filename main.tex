%% Template basiert auf der Vorlage der Uni Graz für VWA: https://latex.tugraz.at/vorlagen/allgemein

%% Versionen:
%% V1: 9. Augugst 2021 (GreiA)
%% V2  25.September 2021 (GreiA) : Fehler mit oldfonts bei der bibtex-Erstellung

\input{template/main_settings}

%% ========================================================================
%% Document metadata: DIESE WERTE BITTE ANPASSEN, wie werden dann automatisch auf der
%% Titelseite angezeigt
%% ========================================================================
\usepackage[T1]{fontenc}
\usepackage[utf8]{inputenc}

\lstset{
  columns=fullflexible,
  keepspaces=true,
  upquote=true,
  literate=
    {—}{{---}}1
    {–}{{--}}1
    {→}{{$\rightarrow$}}1
    {−}{{-}}1
    {…}{{\dots}}1
    {’}{{'}}1
    {“}{{``}}1
    {”}{{''}}1
    {­}{{}}1
}

\newcommand{\mytitle}{Tapyre} 
\newcommand{\mysubtitle}{Entwicklung eines KI-integrierten Produktivitätstools mit Plugin-System}
\newcommand{\myinstitute}{Abteilung für Wirtschaftsingenieure/Betriebsinformatik} 
\newcommand{\mysubmissionyear}{2026} %% Einreich - Jahr
\newcommand{\mysubmissionmonth}{April} %% Monat der Einreichung
\newcommand{\myauthor}{Christian Vorhofer\\Raphael Ladinig}  %% Autoren. Bitte mit \\ Trennen wenn mehrere
\newcommand{\mysupervisor}{GREINÖCKER Albert, Mag. Dr. DI}  %%Betreuer. Bitte mit \\ Trennen wenn mehrere
%\newcommand{\myprojectpartner}{TO REMOVE}  %% Partnerfirma

\newcommand{\mysubject}{SUBJECT}  %% also used for PDF metadata (hyperref)
\newcommand{\mykeywords}{KEYWORDS}  %% also used for PDF metadata (hyperref)


%% header settings
\usepackage{lastpage}

\ohead{\headmark }
\ihead*{\includegraphics[width=3cm]{figures/htl-logo}}
\ifoot{\thepage}  %Will man Anzahl Seiten: /\pageref{LastPage}
\ofoot{\myauthor}


%% ========================================================================
%%%% MISC command definitions
%% ========================================================================
\input{template/mycommands}

%% ========================================================================
%%%% Typographic settings
%% ========================================================================
\input{template/typographic_settings}
\input{template/listing_format}

%% ========================================================================
%%%% MISC usepackages
%% ========================================================================

%% ... it's OK to put here your own usepackage commands ...




%% ========================================================================
%%%% MISC self-defined commands and settings
%% ========================================================================

%% ... it's OK to put here your own newcommand/newenvironment-definitions ...





\hyphenation{ex-am-ple hy-phen-ate}  %% in order to use German umlauts
%% here (Ver-\"of-fent-li-chung), you have to check for
%% activated \usepackage[T1]{fontenc} in the preamble

%% override default language of babel: (be sure to know, what you're
%% doing here)
%\selectlanguage{american}
\selectlanguage{ngerman}

%% ========================================================================
%% bibtex für die Literaturverwaltung: Hier wird der Zitier-Stil festgelegt
%% ========================================================================
\usepackage{natbib} % 
\bibliographystyle{agsm} 


\input{template/pdf_settings}  %% should be *last* definitions in preamble!

%% ========================================================================
%%%% begin{document}
%% ========================================================================

\begin{document} 
\frontmatter                    %% KOMA: roman page numbers and such; only available in scrbook

%% \input{colophon}                %% defines information about editor, LaTeX, font, ...

%% Choose your desired title page:
\input{\mytitlepage}            %% include title page

% \input{template/lock_flag} % Wenn kein Sperrvermerk gemacht werden soll, dann diesen Import einfach auskommentieren

%%\input{template/declaration_TU_Graz}  %% Statutory Declaration
% \input{thanks}                %% this is a suggestion: you have to create this file on demand
% \input{foreword}              %% this is a suggestion: you have to create this file on demand


%% include the abstract without chapter number but include it on table of contents:
%%\let\cleardoublepage\clearpage
\phantomsection
\addcontentsline{toc}{chapter}{Abstract}
   
\include{content/abstract}              %% Abstract
\input{template/affirmation} %%EIDESSTATTLICHE ERKLÄRUNG  

\tableofcontents                %% this produces the table of contents - you might have guessed :-)



%% if myaddlistoftodos is set to "true", the current list of open todos is added:
\ifthenelse{\boolean{myaddlistoftodos}}{
  \newpage\listoftodos          %% handy if you are using todonotes with \todo{}
}{}                             %% with todonotes-package option "disable" you can get rid of any todo in the output

\mainmatter                     %% KOMA: marks main part using arabic page numbers and such; only available in scrbook


%% HIER DIE EIGENEN KAPITEL EINFÜGEN
%\input{content/einleitung}  
%\input{content/latex_beispiele} % Einfach auskommentieren für die tatsächliche Arbeit
%\input{content/mathematik}

\chapter{Einführung in Neuronale Netzwerke}

\section{Künstliche Neuronen}

Künstliche Neuronen bilden die Grundbausteine moderner neuronaler Netze und orientieren sich konzeptionell am Funktionsprinzip biologischer Nervenzellen. Ein künstliches Neuron erhält mehrere Eingangswerte $x_1, x_2, \dots, x_n$, die jeweils mit Gewichten $w_1, w_2, \dots, w_n$ multipliziert werden. Zusammen mit einem Bias-Term $b$ entsteht die gewichtete Summe

\[
z = \sum_{i=1}^{n} w_i x_i + b.
\]

Um dem Modell die Fähigkeit zu geben, nichtlineare Zusammenhänge zu lernen, wird auf diese Summe eine Aktivierungsfunktion angewendet. Typische Aktivierungsfunktionen sind die Sigmoid-Funktion, die Tanh-Funktion oder im modernen Deep Learning vor allem die \textit{Rectified Linear Unit} (ReLU). Das resultierende Ausgabe-Signal des Neurons lautet somit

\[
y = \sigma(z).
\]

Durch die Verschachtelung vieler solcher Neuronen in mehreren Schichten (sogenannten Layers) können sehr komplexe Funktionen modelliert werden. Das „Wissen“ eines neuronalen Netzes ist dabei in den Gewichten und Bias-Werten gespeichert, die während des Trainingsprozesses mithilfe von Optimierungsverfahren wie dem Gradientenabstieg angepasst werden.

Die Fähigkeit eines einzelnen Neurons, lineare Entscheidungsgrenzen zu modellieren, wurde bereits früh durch das Perzeptron-Modell demonstriert. Erst durch die Kombination vieler Neuronen in tieferen Netzen wurde es möglich, hochkomplexe Muster wie Bildmerkmale oder sprachliche Zusammenhänge effizient zu verarbeiten. Künstliche Neuronen stellen somit die Grundlage aller modernen Deep-Learning-Architekturen dar, aus denen später fortgeschrittene Modelle wie Convolutional Neural Networks (CNNs), Rekurrente Neuronale Netze (RNNs) und Transformer hervorgegangen sind (vgl. \cite{Goodfellow-et-al-2016}).

Für diese Arbeit sind künstliche Neuronen besonders relevant, da sie die elementare Recheneinheit der verwendeten Transformer- und Embedding-Modelle darstellen. Alle später beschriebenen Verfahren zur semantischen Repräsentation wissenschaftlicher Texte basieren letztlich auf der Kombination und Optimierung dieser einfachen Bausteine.

\section{Feed-Forward Neural Networks (FNN)}

Ein Feed-Forward Neural Network (FNN) ist die einfachste Form eines neuronalen Netzes und bildet die Grundlage vieler moderner Deep-Learning-Modelle. Der Name beschreibt die zentrale Eigenschaft dieser Architektur: Informationen fließen ausschließlich in eine Richtung, nämlich vom Eingang (\textit{Input Layer}) über eine oder mehrere verdeckte Schichten (\textit{Hidden Layers}) zum Ausgang (\textit{Output Layer}). Rückkopplungen oder Schleifen sind nicht vorhanden.

Ein FNN besteht aus mehreren künstlichen Neuronen, die schichtweise miteinander verbunden sind. Jedes Neuron berechnet aus seinen Eingaben eine gewichtete Summe und wendet anschließend eine Aktivierungsfunktion wie ReLU, Sigmoid oder Tanh an. Dadurch ist das Netzwerk in der Lage, auch komplexe und nichtlineare Zusammenhänge zu modellieren.

Das Training eines FNNs erfolgt mittels des Verfahrens der \textit{Backpropagation}. Dabei wird zunächst der Fehler zwischen der vorhergesagten Ausgabe des Netzes und dem tatsächlichen Zielwert berechnet. Anschließend wird dieser Fehler schrittweise durch das Netzwerk zurückpropagiert, um die Gewichte so anzupassen, dass der Fehler in zukünftigen Durchläufen minimiert wird (vgl. \cite{Goodfellow-et-al-2016}).

Obwohl FNNs im Vergleich zu neueren Architekturen wie CNNs, RNNs oder Transformern relativ einfach aufgebaut sind, bilden sie das Fundament des Deep Learning. Viele moderne Modelle – einschließlich Transformer – lassen sich als Weiterentwicklungen dieses grundlegenden Prinzips verstehen, bei denen zusätzliche Mechanismen wie Attention oder spezielle Schichttypen eingeführt wurden (vgl. \cite{Goodfellow-et-al-2016}).

In dieser Arbeit dienen FNNs vor allem als konzeptionelle Grundlage, um den Übergang von einfachen neuronalen Netzen zu komplexeren Architekturen wie Transformern nachvollziehbar zu machen.

\section{Rekurrente Neuronale Netze (RNN, LSTM)}

Rekurrente Neuronale Netze (RNNs) wurden entwickelt, um sequenzielle Daten wie Texte, Sprache, Musik oder Zeitreihen zu verarbeiten. Im Gegensatz zu Feed-Forward- oder Convolutional-Netzen besitzen RNNs Rückkopplungen, sodass Informationen aus vorherigen Zeitschritten in die Verarbeitung aktueller Eingaben einfließen können. Dadurch verfügen RNNs über eine Art internes „Gedächtnis“.

Ein einfaches RNN kombiniert zu jedem Zeitschritt den aktuellen Eingabewert mit dem vorherigen internen Zustand. Dieses Verfahren eignet sich gut für kurze Sequenzen, stößt jedoch bei längeren Abhängigkeiten an seine Grenzen. Ursache hierfür ist das sogenannte \textit{Vanishing Gradient Problem}, bei dem Gradienten während des Trainings stark abnehmen und relevante Informationen verloren gehen.

Zur Lösung dieses Problems wurden Long Short-Term Memory Netze (LSTMs) entwickelt. LSTMs verfügen über spezielle Schaltelemente, sogenannte \textit{Gates}, die steuern, welche Informationen gespeichert, weitergegeben oder verworfen werden. Dadurch sind LSTMs in der Lage, relevante Informationen über längere Zeiträume hinweg zu behalten und stabiler zu trainieren als klassische RNNs (vgl. \cite{long-short-term-memory}).

LSTMs wurden über viele Jahre erfolgreich in der Sprachverarbeitung eingesetzt, etwa für maschinelle Übersetzung oder Textklassifikation. In modernen NLP-Systemen werden sie jedoch zunehmend durch Transformer-Modelle ersetzt, da diese effizienter trainierbar sind und besser mit sehr langen Texten umgehen können. Für diese Arbeit sind RNNs und LSTMs daher insbesondere im historischen Kontext relevant (vgl. \cite{long-short-term-memory}).

\section{Die Transformer-Architektur}

Transformer-Modelle wurden im Jahr 2017 mit dem Paper \textit{Attention Is All You Need} vorgestellt und haben die natürliche Sprachverarbeitung grundlegend verändert (vgl. \cite{Attention-Is-All-You-Need}). Im Gegensatz zu RNNs und LSTMs verarbeiten Transformer die Eingabe nicht sequenziell, sondern betrachten alle Token eines Textes gleichzeitig. Dies ermöglicht eine effiziente Parallelisierung auf modernen GPUs und verbessert den Umgang mit langen Texten erheblich.

Das zentrale Konzept der Transformer-Architektur ist die sogenannte Self-Attention. Dieser Mechanismus erlaubt es dem Modell, zu bewerten, welche Wörter eines Satzes für die Bedeutung eines anderen Wortes besonders relevant sind. Dadurch können auch weit entfernte Abhängigkeiten innerhalb eines Textes effektiv modelliert werden (vgl. \cite{Attention-Is-All-You-Need}).

Ein weiterer wichtiger Bestandteil ist die Multi-Head Attention. Dabei werden mehrere Attention-Mechanismen parallel eingesetzt, die jeweils unterschiedliche Arten von Beziehungen erfassen können, etwa syntaktische, semantische oder thematische Zusammenhänge. Die Ergebnisse dieser Attention-Köpfe werden anschließend zusammengeführt (vgl. \cite{Attention-Is-All-You-Need}).

Da Transformer keine inhärente Reihenfolge der Eingabe besitzen, werden sogenannte Positional Encodings verwendet. Diese kodieren die Position eines Tokens im Text und werden den Eingaberepräsentationen hinzugefügt, sodass das Modell die Struktur der Sequenz berücksichtigen kann (vgl. \cite{Attention-Is-All-You-Need}).

Ein klassischer Transformer besteht aus einem Encoder und einem Decoder. Während der Encoder den Eingabetext in eine semantisch aussagekräftige Repräsentation überführt, dient der Decoder der Textgenerierung. In vielen modernen Anwendungen – darunter auch die in dieser Arbeit betrachteten Such- und Embedding-Modelle – wird ausschließlich der Encoder verwendet (vgl. \cite{Attention-Is-All-You-Need}).

Transformer bilden die Grundlage der in diesem Projekt eingesetzten Sprach- und Embedding-Modelle und sind daher zentral für das Verständnis der folgenden Kapitel (vgl. \cite{Attention-Is-All-You-Need}).

\section{Bedeutung von Transformern für LLMs und Embeddings}

Transformer-Modelle sind heute die Basis nahezu aller modernen Anwendungen der natürlichen Sprachverarbeitung. Sie bilden das Fundament großer Sprachmodelle (\textit{Large Language Models, LLMs}) sowie leistungsfähiger Embedding-Modelle (vgl. \cite{Attention-Is-All-You-Need,brown2020language,touvron2023llama}).

Durch den Einsatz von Self-Attention können Transformer auch weit entfernte Abhängigkeiten innerhalb eines Textes erfassen. Dies ist insbesondere bei langen oder komplexen Dokumenten von entscheidender Bedeutung, wie sie etwa in wissenschaftlichen Publikationen vorkommen. Frühere Architekturen wie RNNs oder LSTMs konnten solche Zusammenhänge nur eingeschränkt abbilden (vgl. \cite{Attention-Is-All-You-Need}).

Für Embedding-Modelle bieten Transformer den Vorteil, dass sie nicht nur einzelne Wörter, sondern die semantische Bedeutung ganzer Sätze oder Dokumente in dichten Vektorrepräsentationen erfassen. Transformer-Encoder eignen sich daher besonders für Aufgaben wie semantische Suche, Textklassifikation oder Empfehlungssysteme (vgl. \cite{nandakumar2023specter2}).

Moderne Embedding-Modelle wie \textit{SPECTER2}, \textit{Sentence-BERT} oder \textit{E5} basieren auf Transformer-Encodern. Sie ermöglichen es, inhaltlich ähnliche Texte im Vektorraum nahe beieinander abzulegen, während thematisch unterschiedliche Dokumente klar getrennt sind. Diese Eigenschaft ist essenziell für die semantische Suche, wie sie im Rahmen dieses Projekts zur wissenschaftlichen Literatursuche eingesetzt wird (vgl. \cite{nandakumar2023specter2}).

Zusammenfassend lässt sich festhalten, dass Transformer die technologische Grundlage moderner Sprachverarbeitung darstellen. Ohne diese Architektur wären leistungsfähige LLMs sowie präzise Embedding-Modelle nicht realisierbar, wodurch auch das in dieser Arbeit entwickelte Suchsystem in dieser Form nicht möglich wäre (vgl. \cite{Attention-Is-All-You-Need,brown2020language,touvron2023llama,nandakumar2023specter2}).

\chapter{Einführung in Natural Language Processing (NLP)}

Natural Language Processing (NLP) ist ein zentraler Bereich der Künstlichen Intelligenz, der sich mit der automatischen Verarbeitung menschlicher Sprache beschäftigt. Ziel ist es, Texte so zu analysieren und zu interpretieren, dass Computer sprachbasierte Aufgaben ausführen können – etwa Suchanfragen beantworten, Texte zusammenfassen oder Dokumente klassifizieren. Moderne Systeme wie Suchmaschinen, Chatbots oder Sprachassistenten bauen maßgeblich auf Methoden des NLP auf \cite{jurafsky2023slp}.

Während frühe Ansätze vor allem statistische Modelle nutzten, basiert das heutige NLP überwiegend auf tiefen neuronalen Netzen. Eine entscheidende Rolle spielt dabei die Frage, wie Bedeutungen mathematisch repräsentiert werden können. Diese Repräsentationen werden als Embeddings bezeichnet.

\section{Klassische NLP-Ansätze}

Vor dem Aufkommen neuronaler Modelle wurden Texte meist mithilfe statistischer Verfahren dargestellt. Typische Beispiele sind Bag-of-Words, TF--IDF und N-Gramme. Diese Methoden berücksichtigen jedoch weder semantische Beziehungen noch Kontextinformationen. So wird nicht erkannt, dass „Auto“ und „Fahrzeug“ ähnliche Bedeutungen haben oder dass „Bank“ sowohl ein Sitzmöbel als auch ein Finanzinstitut bezeichnen kann \cite{jurafsky2023slp}. 

Für einfache Klassifikationsaufgaben sind solche Modelle oft ausreichend, stoßen jedoch bei komplexeren Anwendungen – etwa semantischer Suche oder Übersetzung – schnell an ihre Grenzen.

\section{Einführung in Embeddings}

Da Computer ausschließlich mit numerischen Daten arbeiten, müssen Texte in Zahlen überführt werden. Embeddings lösen dieses Problem, indem sie Wörter, Sätze oder ganze Dokumente als Vektoren in einem kontinuierlichen Raum darstellen. Dabei gilt:

\begin{itemize}
    \item Ähnliche Bedeutungen sollen ähnliche Vektoren besitzen.
    \item Unterschiedliche Bedeutungen sollen weit voneinander entfernt liegen.
    \item Kontextinformationen sollen möglichst berücksichtigt werden.
\end{itemize}

Embeddings bilden die Grundlage vieler moderner NLP-Systeme und ermöglichen semantische Ähnlichkeitsanalysen sowie inhaltliche Textvergleiche.

\section{Word Embeddings: Word2Vec und GloVe}

Einen bedeutenden Fortschritt stellten Word Embeddings wie Word2Vec dar. Diese Modelle ordnen jedem Wort einen festen Vektor zu und basieren auf der Idee, dass Wörter, die in ähnlichen Kontexten auftreten, ähnliche Repräsentationen erhalten. Dadurch entstehen semantische Strukturen wie:

\[
\mathrm{Koenig} - \mathrm{Mann} + \mathrm{Frau} \approx \mathrm{Koenigin}
\]

Word2Vec \cite{mikolov2013word2vec} und ähnliche Ansätze erfassen grundlegende semantische Beziehungen, ignorieren jedoch die Mehrdeutigkeit von Wörtern: Das Wort „Bank“ hat stets denselben Vektor, unabhängig vom Kontext.

\section{Kontextualisierte Embeddings}

Mit tiefen neuronalen Netzen entstanden Modelle, die Wortbedeutungen kontextabhängig repräsentieren. Ein Beispiel dafür ist ELMo, das für jedes Wort unterschiedliche Vektoren erzeugt – abhängig vom Satz, in dem es vorkommt. Diese Ansätze bilden eine Übergangsphase zwischen klassischen Embeddings und modernen Transformer-Modellen.

\section{Embeddings mit der Transformer-Architektur}

Mit der Einführung der Transformer-Architektur wurden neue Maßstäbe gesetzt. Transformer-Encoder wie BERT \cite{devlin2018bert} oder wissenschaftsspezifische Modelle wie SPECTER \cite{cohan2020specter} erzeugen hochqualitative, kontextualisierte Embeddings, indem sie den gesamten Satz oder sogar das gesamte Dokument berücksichtigen.

Dies führt zu:

\begin{itemize}
    \item kontextabhängigen Wortvektoren,
    \item Satz- und Dokumentrepräsentationen als einzelne Vektoren,
    \item präziser semantischer Modellierung,
    \item robuster Erfassung langer und komplexer Zusammenhänge.
\end{itemize}

Transformer-Embeddings sind daher besonders gut geeignet, um wissenschaftliche Texte mit ihren komplexen Begrifflichkeiten und Strukturmerkmalen zu verarbeiten.

\section{Relevanz für Tapyre Paper Search}

Im Projekt \textit{Tapyre Paper Search} dienen Embeddings dazu, wissenschaftliche Artikel in einem Vektorraum abzubilden. Dokumente mit thematischen Ähnlichkeiten liegen darin räumlich nahe beieinander, was eine präzise semantische Suche ermöglicht. Spezialisierte Modelle wie SPECTER2, die auf wissenschaftlichen Publikationen trainiert wurden, verbessern die Erkennung fachlicher Zusammenhänge nochmals deutlich.


\chapter{Einführung in Agentic AI}

Agentic AI beschreibt ein modernes Paradigma der künstlichen Intelligenz, bei dem Modelle nicht nur reaktiv auf Eingaben antworten, sondern eigenständig Ziele verfolgen, Handlungen planen, Werkzeuge einsetzen und komplexe Aufgaben in mehreren Schritten ausführen. Ein \textit{Agent} ist dabei ein KI-System, das Informationen beschafft, Entscheidungen trifft, Aktionen ausführt und auf Basis neuer Beobachtungen sein weiteres Vorgehen anpasst (vgl. \cite{yao2022react}).

Im Gegensatz zu klassischen Sprachmodellen, die primär als textbasierte Antwortgeneratoren fungieren, agieren agentische Systeme proaktiv und interaktiv. Sie kombinieren Sprachverarbeitung mit Planungs-, Entscheidungs- und Ausführungsmechanismen und können dadurch komplexe Workflows selbstständig bearbeiten.

\section{ReAct: Reasoning + Acting}

Ein grundlegender Ansatz innerhalb von Agentic AI ist das ReAct-Framework (Reasoning + Acting). Dieser Ansatz kombiniert explizite Gedankenschritte (\textit{Reasoning}) mit konkreten Aktionen (\textit{Acting}). ReAct wurde von Yao et al.\ vorgestellt und zielt darauf ab, Entscheidungsprozesse transparent mit tatsächlichen Handlungen zu verbinden (vgl. \cite{yao2022react}).

Ein typischer ReAct-Agent arbeitet in einer zyklischen Struktur:

\begin{enumerate}
    \item Der Agent formuliert einen Gedankenschritt (\textit{Thought}), um das weitere Vorgehen zu planen.
    \item Er führt eine Aktion aus, beispielsweise eine Suchanfrage oder API-Abfrage.
    \item Er erhält eine Beobachtung (\textit{Observation}) als Ergebnis der Aktion.
    \item Auf Basis dieser Information plant er den nächsten Schritt.
\end{enumerate}

Diese Schleife ermöglicht es Agenten, komplexe Aufgaben in überschaubare Teilschritte zu zerlegen und flexibel auf neue Informationen zu reagieren.

\section{Tool Usage}

Ein zentrales Merkmal agentischer Systeme ist die Fähigkeit, externe Werkzeuge (\textit{Tools}) zu verwenden. Diese Tools erweitern die Fähigkeiten eines Sprachmodells erheblich, da sie den Zugriff auf externe Datenquellen und Rechenressourcen ermöglichen. Typische Beispiele sind:

\begin{itemize}
    \item Datenbanken (z.\,B. Qdrant, MySQL),
    \item Web-APIs (z.\,B. arXiv oder Semantic Scholar),
    \item Dateisysteme,
    \item Suchfunktionen,
    \item Ausführung von Python-Skripten.
\end{itemize}

Der Agent wählt ein geeignetes Tool aus, übergibt die notwendigen Parameter, interpretiert die Ergebnisse und nutzt diese als Grundlage für weitere Entscheidungen. Dadurch fungiert das Sprachmodell als eine Art Steuerinstanz, die verschiedene Systeme orchestriert.

Im Kontext dieser Arbeit ermöglicht Tool Usage beispielsweise den Zugriff auf Desktop-Apps.

\section{Model Context Protocol (MCP)}

Das Model Context Protocol (MCP) ist ein offener Standard, der definiert, wie KI-Modelle sicher, strukturiert und nachvollziehbar mit Tools und externen Datenquellen interagieren können. MCP legt unter anderem fest:

\begin{itemize}
    \item wie Tools formal beschrieben werden,
    \item wie Kontextinformationen an Modelle übergeben werden,
    \item wie Modelle Aktionen anfordern,
    \item und wie Ergebnisse standardisiert zurückgegeben werden.
\end{itemize}

Der von OpenAI vorgestellte Standard erleichtert die Integration agentischer Systeme in bestehende Software-Architekturen und reduziert den Implementierungsaufwand komplexer Pipelines (vgl. \cite{openai2024mcp}). Für agentische Systeme wie Tapyre bietet MCP eine strukturierte Grundlage zur Anbindung externer Dienste.

\section{Agent-to-Agent Kommunikation}

In umfangreicheren agentischen Systemen arbeiten häufig mehrere Agenten gemeinsam an einer Aufgabe. Diese sogenannte Agent-to-Agent-Kommunikation ermöglicht eine Aufteilung komplexer Problemstellungen auf spezialisierte Rollen. Beispiele hierfür sind:

\begin{itemize}
    \item ein Analyse-Agent zur Extraktion relevanter Informationen,
    \item ein Recherche-Agent zur Suche nach geeigneten Quellen,
    \item ein Planungs-Agent zur Koordination des Workflows,
    \item ein Evaluations-Agent zur Überprüfung der Ergebnisse.
\end{itemize}

Durch diese Spezialisierung steigt sowohl die Robustheit als auch die Skalierbarkeit des Gesamtsystems. Fehler oder Unsicherheiten einzelner Agenten können durch andere Agenten ausgeglichen werden.

\section{RAG: Retrieval-Augmented Generation}

Retrieval-Augmented Generation (RAG) verbindet Sprachmodelle mit externem, strukturiertem Wissen. Anstatt Antworten ausschließlich aus dem internen Modellwissen zu generieren, ruft ein RAG-System zunächst relevante Dokumente ab und erzeugt darauf aufbauend eine fundierte Antwort. Der Ansatz wurde von Lewis et al.\ eingeführt (vgl. \cite{lewis2020rag}).

Ein typischer RAG-Agent arbeitet in zwei Phasen:

\begin{enumerate}
    \item \textbf{Retrieval}: Suche nach relevanten Dokumenten, beispielsweise über eine Vektorsuche in einer Datenbank wie Qdrant.
    \item \textbf{Generation}: Erzeugung einer Antwort unter Einbeziehung der abgerufenen Inhalte.
\end{enumerate}

Für Tapyre Paper Search ist dieser Ansatz essenziell, da wissenschaftliche Dokumente zunächst semantisch identifiziert und anschließend analysiert, zusammengefasst oder weiterverarbeitet werden.

\section{Multi-Agent Systems}

Multi-Agent Systems (MAS) bestehen aus mehreren autonomen, oft spezialisierten Agenten, die parallel oder kooperativ zusammenarbeiten. Solche Systeme bieten mehrere Vorteile:

\begin{itemize}
    \item höhere Robustheit durch Aufgabenverteilung,
    \item bessere Skalierbarkeit,
    \item parallele Bearbeitung komplexer Probleme,
    \item gezielte Spezialisierung einzelner Agenten.
\end{itemize}

Multi-Agent-Systeme werden zunehmend in Forschung, Retrieval-Systemen und komplexen KI-Anwendungen eingesetzt. Auch für zukünftige Erweiterungen von Tapyre Paper Search bieten sie eine flexible und leistungsfähige Architekturgrundlage.

\chapter{Grundkonzepte der verwendeten Technologien}

Für die Umsetzung von Tapyre Paper Search werden mehrere moderne Software- und Infrastrukturtechnologien eingesetzt, die zusammen eine performante und erweiterbare Architektur ermöglichen. Dieses Kapitel erläutert die technischen Grundlagen und zeigt jeweils den praktischen Bezug zur Umsetzung des Systems (z.\,B. Deployment, Datenhaltung, semantische Suche, API-Schicht und GPU-beschleunigte Modellinferenz).

\section{Docker und Containerisierung}

Docker ist eine Plattform zur Containerisierung von Anwendungen (vgl. \cite{docker2025security}). Im Gegensatz zu klassischen virtuellen Maschinen teilen sich Container den Kernel des Host-Betriebssystems, sind aber logisch voneinander isoliert. Diese Isolation wird durch mehrere Linux-Technologien erreicht, insbesondere Namespaces, Control Groups (cgroups) und Union-Filesystems (vgl. \cite{docker2025security,container_architecture_namespaces}):

\begin{itemize}
    \item \textbf{Namespaces}: isolieren Prozesse, Netzwerk, Benutzer und Dateisysteme.
    \item \textbf{Control Groups (cgroups)}: begrenzen CPU-, RAM- und I/O-Ressourcen.
    \item \textbf{Union-Filesystems} (z.\,B. OverlayFS): ermöglichen effiziente, layer-basierte Images.
\end{itemize}

Docker ermöglicht reproduzierbare Umgebungen, schnelle Deployments und konsistente Konfigurationen (vgl. \cite{docker2025security}). Für Tapyre Paper Search ist dies insbesondere relevant, da Komponenten wie Qdrant, MySQL und die Python-Dienste unabhängig voneinander, aber dennoch in einer einheitlichen Laufzeitumgebung betrieben werden können. Dadurch lassen sich Entwicklungs-, Test- und Produktionsumgebungen konsistent abbilden und Abhängigkeiten (z.\,B. Datenbankversionen) kontrollierbar halten.

\section{MySQL als relationale Datenbank}

MySQL ist ein relationales Datenbanksystem, das Daten strukturiert in Tabellen speichert. Das Datenmodell folgt einem relationalen Schema, bei dem Entitäten über Primär- und Fremdschlüssel miteinander verbunden sind. Für die interne Datenorganisation nutzt MySQL (InnoDB) insbesondere B\textsuperscript{+}-Bäume, die als Clustered und Secondary Indexes realisiert sind (vgl. \cite{mysql_innodb_indexes,percona2024mysqlindexes}):

\begin{itemize}
    \item \textbf{Clustered Index}: InnoDB speichert Daten entlang des Primärschlüssels; die Tabelle ist selbst als B\textsuperscript{+}-Baum organisiert.
    \item \textbf{Secondary Indexes}: weitere B\textsuperscript{+}-Bäume, die Verweise auf die Primärschlüsselzeilen enthalten.
    \item \textbf{Effiziente Suche}: durch die logarithmische Höhe der B\textsuperscript{+}-Bäume können Suchanfragen performant ausgeführt werden.
\end{itemize}

MySQL bietet außerdem ACID-Transaktionen, die Datenkonsistenz und Zuverlässigkeit sicherstellen (vgl. \cite{gray1992transaction}):

\begin{itemize}
    \item \textbf{Atomicity}: Eine Transaktion wird entweder vollständig ausgeführt oder verworfen.
    \item \textbf{Consistency}: Integritätsregeln bleiben erhalten.
    \item \textbf{Isolation}: Parallel ausgeführte Transaktionen beeinflussen sich kontrolliert bzw. entsprechend des Isolation-Levels nicht unzulässig.
    \item \textbf{Durability}: bestätigte Änderungen bleiben dauerhaft gespeichert.
\end{itemize}

In Tapyre Paper Search wird MySQL zur Speicherung strukturierter Metadaten eingesetzt (z.\,B. Titel, Autoren, Kategorien, Importstatus). Das ermöglicht klassische Filter- und Verwaltungsabfragen (z.\,B. nach Datum, Kategorie oder Verarbeitungsstatus), während die semantische Suche separat über die Vektordatenbank erfolgt.

\section{Qdrant und Approximate Nearest Neighbor Search}

Qdrant ist eine spezialisierte Datenbank zur Speicherung und Suche von Vektorrepräsentationen (Embeddings) und ist für semantische Suchanwendungen optimiert (vgl. \cite{qdrant_docs}). Im Gegensatz zu relationalen Datenbanken steht hier nicht die textbasierte Volltextsuche im Vordergrund, sondern die Ähnlichkeitssuche in einem Vektorraum, in dem Dokumente durch numerische Repräsentationen beschrieben werden.

\subsection{Speicherstruktur}

Eine Qdrant-Collection besteht typischerweise aus Vektoren, Payload-Daten (Metadaten) sowie internen Segmenten zur Datenorganisation (vgl. \cite{qdrant_collections}):

\begin{itemize}
    \item \textbf{Vektoren} (häufig 768 oder 1024 Dimensionen, abhängig vom Embedding-Modell),
    \item \textbf{Payload} wie Titel, DOI, Autoren oder Jahr,
    \item \textbf{Segmente} zur internen Aufteilung der Daten.
\end{itemize}

Diese Struktur ist auf schnelle Ähnlichkeitsabfragen und effiziente Datenzugriffe ausgelegt (vgl. \cite{qdrant_similarity}). In Tapyre Paper Search erlaubt die Payload zudem das nachträgliche Filtern semantischer Treffer (z.\,B. nach Jahr oder Kategorie), während die eigentliche Ranking-Logik über Vektorsimilarität erfolgt.

\subsection{Approximate Nearest Neighbor (ANN)}

Da ein exakter Vergleich aller Vektoren bei großen Datenmengen ineffizient ist, verwenden Vektordatenbanken ANN-Algorithmen. Qdrant nutzt hierfür u.\,a. den HNSW-Index (Hierarchical Navigable Small World), einen graphbasierten ANN-Algorithmus (vgl. \cite{malkov2020hnsw,qdrant_indexing,qdrant_hnsw_course}):

\begin{itemize}
    \item mehrschichtige Graphstruktur,
    \item wenige Knoten auf höheren Ebenen (grobe Orientierung),
    \item viele Knoten auf unteren Ebenen (feine Suche),
    \item Navigation vom groben in den feinen Suchraum,
    \item schnelle Annäherung an die nächsten Nachbarn.
\end{itemize}

HNSW kombiniert hohe Geschwindigkeit mit hoher Treffergenauigkeit und eignet sich besonders für semantische Suchsysteme (vgl. \cite{malkov2020hnsw,qdrant_similarity}). In Tapyre Paper Search ist dies zentral, da auch größere Paper-Sammlungen interaktiv durchsuchbar bleiben sollen.

\subsection{Ähnlichkeitsmaße}

Qdrant unterstützt verschiedene Distanz- bzw. Ähnlichkeitsmaße, darunter Kosinusähnlichkeit, euklidische Distanz und skalares Produkt (vgl. \cite{qdrant_similarity}):

\begin{itemize}
    \item \textbf{Kosinusähnlichkeit} (häufiger Standard für NLP-Embeddings),
    \item \textbf{Euklidische Distanz},
    \item \textbf{Skalares Produkt} (Dot Product).
\end{itemize}

In Tapyre wird hauptsächlich die Kosinusähnlichkeit eingesetzt, da sie bei normalisierten Text-Embeddings eine robuste Näherung semantischer Ähnlichkeit bietet (vgl. \cite{qdrant_similarity}).

\section{Flask und REST-APIs}

Flask ist ein leichtgewichtiges Webframework für Python und dient in Tapyre zur Bereitstellung von REST-APIs (vgl. \cite{flask_docs}). REST (\textit{Representational State Transfer}) ist ein Architekturstil für verteilte Systeme, der u.\,a. von Fielding beschrieben wurde (vgl. \cite{fielding2000rest}). REST-basierte Schnittstellen verwenden standardisierte HTTP-Methoden:

\begin{itemize}
    \item \textbf{GET}: Abfrage von Daten,
    \item \textbf{POST}: Erstellen neuer Ressourcen bzw. Auslösen von Verarbeitung,
    \item \textbf{PUT/PATCH}: Aktualisieren von Ressourcen,
    \item \textbf{DELETE}: Löschen von Ressourcen.
\end{itemize}

REST-APIs verwenden häufig JSON als Datenaustauschformat und sind zustandslos, d.\,h. jeder Request enthält alle für die Verarbeitung notwendigen Informationen (vgl. \cite{fielding2000rest}).

Ein vereinfachtes Beispiel für eine Anfrage an einen Embedding-Endpunkt:

\begin{verbatim}
POST /embed
{
    "text": "Deep learning improves scientific search."
}
\end{verbatim}

In Tapyre Paper Search dient die API-Schicht als klar definierte Kommunikationsgrenze zwischen Komponenten (z.\,B. Frontend, Orchestrierung und Embedding-Service). Das erleichtert Tests, ermöglicht unabhängige Skalierung einzelner Dienste und reduziert Kopplung zwischen Systemteilen.

\section{PyTorch und GPU-Beschleunigung}

PyTorch ist ein Framework für Deep Learning und wird in Tapyre zur Berechnung von Embeddings eingesetzt (vgl. \cite{paszke2019pytorch}). Da Transformer-Modelle wie SPECTER2 sehr rechenintensiv sind, wird die Ausführung auf GPUs genutzt, um Matrixoperationen massiv zu beschleunigen. CUDA stellt hierfür eine Plattform bereit, um entsprechende Operationen auf der Grafikkarte auszuführen (vgl. \cite{nvidia_cuda_guide}).

Relevante Vorteile von PyTorch sind unter anderem (vgl. \cite{paszke2019pytorch}):

\begin{itemize}
    \item dynamische Rechengraphen,
    \item breite Modell- und Tool-Unterstützung,
    \item nahtlose GPU-Nutzung,
    \item gute Integrierbarkeit in moderne NLP-Pipelines.
\end{itemize}

Für Tapyre Paper Search ist dies praktisch relevant, da die Embedding-Erzeugung typischerweise den größten Rechenanteil der Pipeline darstellt. GPU-Beschleunigung reduziert die Laufzeit der Indexierung und ermöglicht schnellere Aktualisierungen des Paper-Bestands.

\section{Zusammenfassung}

Docker stellt reproduzierbare Umgebungen für die komponentenbasierte Ausführung bereit (vgl. \cite{docker2025security}), MySQL speichert strukturierte Metadaten effizient über Indizes und Transaktionen (vgl. \cite{mysql_innodb_indexes,gray1992transaction}), Qdrant ermöglicht performante semantische Vektorsuche mittels ANN/HNSW (vgl. \cite{qdrant_docs,malkov2020hnsw}), Flask dient als Kommunikationsschicht über REST (vgl. \cite{flask_docs,fielding2000rest}), und PyTorch führt rechenintensive Embedding-Modelle GPU-beschleunigt aus (vgl. \cite{paszke2019pytorch,nvidia_cuda_guide}). Zusammen bilden diese Technologien die technische Grundlage für ein flexibles und leistungsfähiges Informationssystem wie Tapyre Paper Search.

\chapter{Entwicklung von Tapyre als Agentic-AI-System}

In diesem Kapitel wird die Entwicklung von Tapyre als \emph{Agentic AI}-System beschrieben. Im Gegensatz zu klassischen, rein reaktiven Sprachmodellen, die ausschließlich nach dem Prinzip \textit{Input $\rightarrow$ Output} arbeiten, nutzt Tapyre einen Agenten, der eigenständig Entscheidungen trifft, Tools aufruft und Aufgaben iterativ in einem ReAct-Loop (Reasoning + Acting) ausführt (vgl. \cite{yao2022react, schick2023toolformer, wang2024surveyagentic}).

Der Agent fungiert dabei als steuernde Instanz zwischen dem Sprachmodell, externen Werkzeugen und der Systemumgebung. Die Kopplung zwischen Kernsystem, LLM und Plugins ist bewusst lose gehalten, um eine hohe Erweiterbarkeit, Wartbarkeit und Austauschbarkeit einzelner Komponenten zu gewährleisten. Dieses Architekturprinzip orientiert sich an etablierten Entwurfsmustern modularer Softwaresysteme (vgl. \cite{gamma1994designpatterns}).

Im Folgenden werden die wichtigsten Bausteine dieser Architektur erläutert:

\begin{itemize}
  \item abstrakte LLM-Schnittstelle und konkrete Implementierung für Ollama,
  \item der Agent auf Basis von LangChain und dem ReAct-Paradigma,
  \item das Plugin-Konzept und dessen Abbildung auf LangChain-Tools,
  \item dynamisches Laden der Plugins zur Laufzeit,
  \item ein konkretes Beispiel-Plugin (\texttt{AppPlugin}),
  \item lose Kopplung und systemweite Erweiterbarkeit.
\end{itemize}

\section{Abstraktion der LLM-Schnittstelle}

Um Tapyre unabhängig von einem konkreten Sprachmodell oder Anbieter zu halten, wird eine abstrakte LLM-Schnittstelle definiert. Jeder unterstützte LLM-Typ (z.\,B. Ollama oder cloudbasierte APIs) muss lediglich diese Schnittstelle implementieren. Dadurch kann das zugrunde liegende Sprachmodell ausgetauscht werden, ohne dass der Agent oder bestehende Plugins angepasst werden müssen.

Dieses Vorgehen folgt etablierten Architekturprinzipien wie Interface- und Factory-Abstraktionen, die eine klare Trennung von Schnittstelle und Implementierung vorsehen (vgl. \cite{gamma1994designpatterns}).

\lstinputlisting[
style=python,
caption={Abstrakte LLM-Schnittstelle},
captionpos=b
]{sourcecode/tapyre/tapyre/src/abstractions/llm.py}

Die Schnittstelle kapselt die Kommunikation mit dem Sprachmodell vollständig und stellt dem restlichen System eine einheitliche Interaktionsmöglichkeit zur Verfügung.

Eine konkrete Implementierung für das lokale LLM-Framework sieht wie folgt aus:

\lstinputlisting[
style=python,
caption={OllamaLLM als konkrete Implementierung},
captionpos=b
]{sourcecode/tapyre/tapyre/src/implementations/ollama_llm.py}

Der restliche Systemkern arbeitet ausschließlich mit dem Interface \texttt{LLM} und erhält Instanzen von \texttt{BaseChatModel}. Welches konkrete Sprachmodell im Hintergrund verwendet wird, ist für den Agenten und die Plugins vollständig transparent.

\section{Der Agent und der ReAct-Loop}

Der Agent selbst ist ebenfalls als Abstraktion definiert und stellt lediglich eine zentrale Methode \texttt{ask} bereit, die eine Anfrage entgegennimmt und eine Antwort erzeugt:

\lstinputlisting[
style=python,
caption={Abstrakte Agent-Schnittstelle},
captionpos=b
]{sourcecode/tapyre/tapyre/src/abstractions/agent.py}

Diese minimale Schnittstelle verdeutlicht die konzeptionelle Rolle des Agenten: Er fungiert als vermittelnde Instanz zwischen Benutzeranfrage, Sprachmodell und verfügbaren Tools.

Die konkrete Implementierung \texttt{PluginAgent} basiert auf dem Framework LangChain und verwendet den Agententyp \texttt{ZERO\_SHOT\_REACT\_DESCRIPTION}. Dieser Agententyp setzt explizit das ReAct-Paradigma um, bei dem das Modell interne Gedankenschritte (\emph{Reasoning}) mit konkreten Aktionen (\emph{Acting}) kombiniert (vgl. \cite{yao2022react, langchain2023}).

\lstinputlisting[
style=python,
caption={PluginAgent mit ReAct-Agententyp},
captionpos=b
]{sourcecode/tapyre/tapyre/src/implementations/plugin_agent.py}

Zentrale Aspekte dieser Implementierung sind:

\begin{itemize}
  \item \textbf{ReAct-Loop}: Der Agent erzeugt intern eine Sequenz aus \enquote{Thought}, \enquote{Action} und \enquote{Observation}, wodurch Planung und Ausführung explizit miteinander verknüpft werden.
  \item \textbf{Tool-Auswahl}: Die verfügbare Tool-Liste wird dynamisch aus den geladenen Plugins generiert. Das LLM erhält ausschließlich deren Beschreibungen und entscheidet selbstständig, welches Tool geeignet ist.
  \item \textbf{Iterationsbegrenzung}: Die maximale Anzahl von Tool-Aufrufen wird begrenzt, um Endlosschleifen zu vermeiden und die Kontrolle über den Agentenlauf zu behalten (vgl. \cite{schick2023toolformer}).
\end{itemize}

\section{Plugins als Tools: Lose Kopplung durch Interfaces}

Plugins stellen die eigentliche funktionale Erweiterbarkeit des Systems dar, etwa zum Starten von Anwendungen, zur Abfrage externer Datenquellen oder zur Interaktion mit dem Dateisystem. Sie sind über eine abstrakte Basisklasse definiert und können unabhängig vom Kernsystem implementiert werden.

\lstinputlisting[
style=python,
caption={Abstrakte Plugin-Basisklasse},
captionpos=b
]{sourcecode/tapyre/tapyre/src/abstractions/plugin.py}

Wesentliche Eigenschaften dieses Ansatzes sind:

\begin{itemize}
  \item \textbf{Interface-basiertes Design}: Jedes Plugin implementiert lediglich die Methode \texttt{run()}.
  \item \textbf{Lose Kopplung}: Der Agent kennt ausschließlich die daraus erzeugten LangChain-Tools, nicht jedoch die konkrete Plugin-Implementierung.
  \item \textbf{Tool-Integration}: Mithilfe von \texttt{LCTool.from\_function} wird die \texttt{run()}-Methode automatisch als Tool für den ReAct-Loop verfügbar gemacht (vgl. \cite{langchain2023}).
\end{itemize}

\section{Dynamisches Laden der Plugins}

Um neue Funktionalitäten ohne Änderungen am Hauptprogramm integrieren zu können, werden Plugins dynamisch zur Laufzeit geladen. Dieses Vorgehen entspricht gängigen Entwurfsmustern für modulare und erweiterbare Softwaresysteme (vgl. \cite{gamma1994designpatterns}).

\lstinputlisting[
style=python,
caption={Dynamischer PluginLoader},
captionpos=b
]{sourcecode/tapyre/tapyre/src/runtime/plugin_loader.py}

Die Vorteile dieses Ansatzes sind:

\begin{itemize}
  \item pluginbasierte Erweiterbarkeit ohne Neukompilierung,
  \item keine Änderungen am Kernsystem erforderlich,
  \item automatische Erkennung neuer Plugins mittels Reflection und Introspection.
\end{itemize}

\section{Beispiel: AppPlugin zur Steuerung lokaler Anwendungen}

Ein konkretes Beispiel für ein Plugin ist das \texttt{AppPlugin}. Dieses liest installierte Desktop-Anwendungen aus \texttt{.desktop}-Dateien aus und ermöglicht es dem Agenten, Anwendungen auf Benutzeranfrage zu starten.

\lstinputlisting[
style=python,
caption={AppPlugin als konkretes Plugin},
captionpos=b
]{sourcecode/tapyre/tapyre/src/plugins/app_plugin.py}

Das Plugin implementiert die abstrakte Plugin-Schnittstelle und wird dem Agenten als Tool zur Verfügung gestellt. Im ReAct-Loop kann das Sprachmodell selbstständig entscheiden, wann der Aufruf dieses Tools sinnvoll ist (vgl. \cite{yao2022react, langchain2023}).

\section{Zusammenspiel von Agent, Plugins und ReAct-Loop}

Der typische Ablauf einer Anfrage kombiniert mehrere etablierte Konzepte der Agentic AI:

\begin{enumerate}
  \item ReAct-basiertes Reasoning (vgl. \cite{yao2022react}),
  \item LLM-gestützte Tool-Nutzung (vgl. \cite{schick2023toolformer}),
  \item modulare Softwarearchitekturen (vgl. \cite{gamma1994designpatterns}),
  \item agentische Selbstorganisation (vgl. \cite{wang2024surveyagentic}),
  \item optional: multi-agentische Koordination (vgl. \cite{du2023ma, hong2023metagpt}).
\end{enumerate}

Durch diese Architektur wird Tapyre zu einem vollwertigen \emph{Agentic AI}-System. Das Sprachmodell übernimmt die Rolle einer intelligenten Steuerungsinstanz, die eigenständig plant, Tools auswählt und Entscheidungen iterativ weiterentwickelt, während die Plugin-Struktur eine maximale Erweiterbarkeit des Gesamtsystems sicherstellt.

\chapter{Architektur und Implementierung von Tapyre Paper Search}

In diesem Kapitel wird die Architektur und Implementierung von Tapyre Paper Search beschrieben. Das System dient der automatisierten Verarbeitung, Indexierung und semantischen Durchsuchung wissenschaftlicher Publikationen. Ziel ist es, große Mengen an Forschungsarbeiten aus unterschiedlichen Quellen strukturiert aufzubereiten und sowohl klassisch (Metadaten) als auch semantisch (Vektorrepräsentationen) durchsuchbar zu machen.

Die Architektur folgt einem modularen Ansatz mit klar getrennten Verantwortlichkeiten für Datenbeschaffung, Textverarbeitung, Embedding-Erzeugung, Speicherung und Orchestrierung. Dadurch ist das System leicht erweiterbar, wartbar und auf unterschiedliche Datenquellen sowie Embedding-Modelle anpassbar.

Im Folgenden werden die zentralen Bausteine erläutert:

\begin{itemize}
\item abstrakte Kernschnittstellen für Datenquellen, Embeddings und Datenbanken,
\item konkrete Implementierungen für arXiv, Specter2, Qdrant und MySQL,
\item die PDF-Verarbeitung als technische Herausforderung,
\item die Pipeline zur Orchestrierung des Gesamtprozesses,
\item das Zusammenspiel von strukturierter und semantischer Suche.
\end{itemize}

\section{Abstraktion der Datenquellen}

Um unterschiedliche Paper-Quellen einbinden zu können, wird eine abstrakte Schnittstelle für Datenprovider definiert. Diese legt fest, wie neue Dokumente geladen und bereitgestellt werden, ohne den restlichen Verarbeitungsprozess zu beeinflussen.

\lstinputlisting[
style=python,
caption={Abstrakte Schnittstelle für Datenquellen},
captionpos=b
]{sourcecode/tapyre/tapyre-paper-search/src/core/data_provider.py}

Durch diese Abstraktion können neue Datenquellen (z.,B. PubMed oder lokale Archive) ergänzt werden, ohne Änderungen an der Pipeline oder den Datenbankschichten vorzunehmen. Das System folgt damit dem Open-Closed-Prinzip etablierter Softwarearchitekturen.

\section{arXiv als konkrete Datenquelle}

Eine konkrete Implementierung dieser Schnittstelle stellt der arXiv-Datenprovider dar. Er übernimmt das Abrufen von Metadaten sowie das Herunterladen der zugehörigen PDF-Dokumente.

\lstinputlisting[
style=python,
caption={arXivDataProvider zur Anbindung der arXiv-API},
captionpos=b
]{sourcecode/tapyre/tapyre-paper-search/src/impl/arxiv_data_provider.py}

Der Provider verarbeitet unter anderem Titel, Autoren, Abstracts, Kategorien sowie die PDF-URL eines Papers. Durch die Trennung von Metadatenbeschaffung und nachgelagerter Textverarbeitung bleibt die Architektur flexibel gegenüber Änderungen der Datenquelle.

\section{PDF-Verarbeitung und Textextraktion}

Die Umwandlung von wissenschaftlichen PDFs in maschinenlesbaren Text stellt eine zentrale technische Herausforderung dar. Wissenschaftliche Dokumente enthalten häufig mehrspaltige Layouts, Formeln, Fußnoten und Seitenheader, die eine robuste Textextraktion erschweren.

\lstinputlisting[
style=python,
caption={PDF-zu-Text-Konvertierung mit PyMuPDF},
captionpos=b
]{sourcecode/tapyre/tapyre-paper-search/src/impl/fitz_pdf_converter.py}

Die Implementierung basiert auf PyMuPDF (fitz) und extrahiert den Text seitenweise. Diese Lösung stellt einen praktikablen Kompromiss zwischen Performance und Textqualität dar und eignet sich insbesondere für große Paper-Sammlungen.

\section{Abstraktion der Embedding-Erzeugung}

Die semantische Suche erfordert die Umwandlung von Text in hochdimensionale Vektoren. Um unterschiedliche Modelle einsetzen zu können, wird eine abstrakte Embedder-Schnittstelle definiert.

\lstinputlisting[
style=python,
caption={Abstrakte Embedder-Schnittstelle},
captionpos=b
]{sourcecode/tapyre/tapyre-paper-search/src/core/embedder.py}

Diese Abstraktion erlaubt es, verschiedene Embedding-Modelle auszutauschen oder parallel zu evaluieren, ohne Änderungen an der restlichen Systemarchitektur vorzunehmen.

\section{Specter2 als semantisches Embedding-Modell}

Als konkrete Implementierung kommt Specter2 zum Einsatz, ein speziell für wissenschaftliche Texte entwickeltes Transformer-Modell. Es erzeugt Vektoren, die den semantischen Inhalt eines Papers erfassen und sich besonders für wissenschaftliche Suchanwendungen eignen.

\lstinputlisting[
style=python,
caption={Specter2Embedder zur Erzeugung semantischer Vektoren},
captionpos=b
]{sourcecode/tapyre/tapyre-paper-search/src/impl/specter_2_embedder.py}

Durch die Verwendung eines domänenspezifischen Modells wird eine deutlich bessere semantische Repräsentation erzielt als mit generischen Sprachmodellen.

\section{Abstraktion der Datenhaltung}

Das System unterscheidet bewusst zwischen strukturierter Datenhaltung (Metadaten) und semantischer Vektorspeicherung. Eine abstrakte Datenbankschnittstelle definiert dabei die notwendigen Operationen.

\lstinputlisting[
style=python,
caption={Abstrakte Datenbankschnittstelle},
captionpos=b
]{sourcecode/tapyre/tapyre-paper-search/src/core/database.py}

Diese Trennung erlaubt es, relationale und vektorbasierte Datenbanken gezielt für ihre jeweiligen Stärken einzusetzen.

\section{MySQL für strukturierte Metadaten}

Metadaten wie Titel, Autoren, Kategorien und Statusinformationen werden in einer relationalen Datenbank gespeichert. Die konkrete Implementierung basiert auf MySQL.

\lstinputlisting[
style=python,
caption={MySQL-Datenbankanbindung für Paper-Metadaten},
captionpos=b
]{sourcecode/tapyre/tapyre-paper-search/src/impl/mysql_database.py}

Relationale Datenbanken eignen sich besonders für konsistente Speicherung, Filterung und relationale Abfragen auf Metadaten.

\section{Qdrant als Vektordatenbank}

Für die semantische Suche werden die Embeddings in einer spezialisierten Vektordatenbank gespeichert. Qdrant ermöglicht effiziente Approximate-Nearest-Neighbor-Suche auf hochdimensionalen Vektoren.

\lstinputlisting[
style=python,
caption={Qdrant-Datenbank für semantische Suche},
captionpos=b
]{sourcecode/tapyre/tapyre-paper-search/src/impl/qdrant_database.py}

Durch den Einsatz von HNSW-basierten Indexstrukturen können auch sehr große Paper-Sammlungen performant durchsucht werden.

\section{Pipeline zur Orchestrierung des Gesamtprozesses}

Die Pipeline bildet das zentrale Orchestrierungselement des Systems. Sie verbindet Datenquelle, PDF-Verarbeitung, Embedding-Erzeugung und Speicherung zu einem konsistenten Ablauf.

\lstinputlisting[
style=python,
caption={Pipeline zur Verarbeitung und Indexierung von Papers},
captionpos=b
]{sourcecode/tapyre/tapyre-paper-search/src/impl/pipeline.py}

Der Ablauf gliedert sich dabei in folgende Schritte:

\begin{enumerate}
\item Abruf neuer Papers aus der Datenquelle,
\item Download und Textextraktion aus PDFs,
\item Chunking und Embedding der Texte,
\item Speicherung von Metadaten und Vektoren.
\end{enumerate}

Durch die klare Trennung der einzelnen Schritte bleibt die Pipeline leicht erweiterbar und testbar.

\section{Zusammenspiel der Komponenten}

Das Zusammenspiel der beschriebenen Komponenten ermöglicht eine skalierbare und flexible Paper-Search-Plattform. Während relationale Datenbanken effiziente Metadatenabfragen erlauben, stellt die Vektordatenbank eine leistungsfähige semantische Suche bereit. Die modulare Architektur erlaubt es, neue Datenquellen, Modelle oder Datenbanken mit minimalem Implementierungsaufwand zu integrieren.

Damit bildet Tapyre Paper Search die technische Grundlage für eine moderne, agentenfähige Forschungsplattform, die klassische Informationsretrieval-Ansätze mit aktuellen Methoden der semantischen Suche kombiniert.
\chapter{Implementierung der Benutzeroberflächen}

Nachdem in den vorangegangenen Kapiteln die Entwicklung der Backend-Logik, der Agentic-AI-Architektur sowie der Datenverarbeitungspipeline detailliert beschrieben wurde, widmet sich dieses Kapitel der Implementierung der Benutzeroberflächen. 

Es werden zwei unterschiedliche Frontends vorgestellt, die auf die spezifischen Anforderungen der jeweiligen Anwendungsfälle zugeschnitten sind:
\begin{itemize}
    \item Die \textbf{Tapyre Desktop-Anwendung}, die als native Schnittstelle für den AI-Agenten dient und eine tiefe Integration in die Arbeitsumgebung ermöglicht.
    \item Die \textbf{Tapyre Paper Search Webanwendung}, die eine intuitive Suche und Exploration wissenschaftlicher Publikationen im Browser bietet.
\end{itemize}

Das Kapitel beleuchtet dabei die Auswahl der Technologien, die Designentscheidungen sowie die technische Umsetzung der grafischen Oberflächen.

\section{Tapyre}

Tapyre ist eine Desktop-Anwendung, die als grafische Benutzeroberfläche (GUI) für das im vorherigen Kapitel beschriebene Agentic-AI-System dient. Während die Kernlogik des Agenten auf textbasierten Interaktionen aufbaut, stellt diese Anwendung eine intuitive und visuelle Schnittstelle bereit, um die Nutzung des KI-Assistenten im alltäglichen Arbeitsablauf zu erleichtern.

Ein zentrales Ziel bei der Entwicklung war die nahtlose Integration in die bestehende Desktop-Umgebung, um eine konsistente und reibungslose Benutzererfahrung zu gewährleisten. 

\subsection{Auswahl des Technologie-Stacks}

Die Auswahl der geeigneten Technologien für die Tapyre-Desktop-Anwendung basierte auf mehreren zentralen Anforderungen: der nahtlosen Integration der im Backend entwickelten Python-Logik, einer hohen Performance der Benutzeroberfläche sowie einer effizienten und reproduzierbaren Entwicklungsumgebung.

Als Programmiersprache wurde \textbf{Python} gewählt (vgl. \cite{python}). Da der Kern des Agentic-AI-Systems und die genutzten Frameworks bereits in Python implementiert sind, entfällt durch die Verwendung derselben Sprache für das Frontend der Overhead einer sprachübergreifenden Schnittstelle (z.\,B. REST-API oder IPC). Dies ermöglicht eine direkte und performante Kommunikation zwischen GUI und Agentenlogik.

Für die grafische Benutzeroberfläche fiel die Entscheidung auf GTK 3 (vgl. \cite{gtk3}). GTK ist eines der führenden Toolkits für Linux-Desktop-Umgebungen und bietet durch \textbf{PyGObject} exzellente Python-Bindings (vgl. \cite{pygobject}). Es erlaubt die Entwicklung nativer Anwendungen, die sich optisch und funktional harmonisch in den Linux-Desktop einfügen. Ein weiterer entscheidender Vorteil ist die Möglichkeit, das Design der Anwendung mittels \textbf{CSS} (Cascading Style Sheets) anzupassen, was eine flexible und moderne Gestaltung der Oberfläche ermöglicht (vgl. \cite{w3c_css}).

Um die Herausforderungen des Abhängigkeitsmanagements und der Reproduzierbarkeit zu lösen, kommen \textbf{Nix} (vgl. \cite{nixos}) und \textbf{uv} (vgl. \cite{astral_uv}) zum Einsatz. Nix garantiert eine deterministische Systemumgebung, in der auch binäre Abhängigkeiten (wie die GTK-Bibliotheken) zuverlässig bereitgestellt werden. Ergänzend dazu wird uv als extrem schneller Paketmanager für Python verwendet, um Installationszeiten zu minimieren und eine strikte Versionierung der Python-Bibliotheken sicherzustellen.

\subsection{Evaluierung alternativer UI-Toolkits}

Vor der Festlegung auf GTK 3 und Python wurden mehrere alternative Technologien für die Entwicklung der Desktop-Oberfläche evaluiert. Ziel war es, ein Framework zu finden, das eine moderne Optik, hohe Performance und eine nahtlose Anbindung an die Python-basierte Agentenlogik ermöglicht.

\subsubsection{Tauri}
Tauri ist ein modernes Framework, das es ermöglicht, Desktop-Anwendungen mit Web-Technologien (HTML, CSS, JavaScript/TypeScript) für das Frontend und Rust für das Backend zu erstellen (vgl. \cite{tauri}).
\begin{itemize}
    \item \textbf{Vorteile:} Tauri zeichnet sich durch extrem kleine Dateigrößen und hohe Performance aus, da es auf den nativen Webview des Betriebssystems zurückgreift. Das Sicherheitsmodell ist robust, und die Trennung von Frontend und Backend ist strikt.
    \item \textbf{Nachteile:} Die Nutzung von Tauri hätte die Einführung von zwei neuen Technologie-Stacks erfordert: Rust für die Core-Logik und JavaScript/TypeScript für das UI. Da der bestehende Agenten-Kern in Python geschrieben ist, hätte eine komplexe Inter-Process Communication (IPC) oder eine komplette Neuentwicklung in Rust stattfinden müssen. Dies hätte die Komplexität des Projekts unnötig erhöht.
\end{itemize}

\subsubsection{Astal}
Astal ist eine spezialisierte Bibliothek, die es ermöglicht, GTK-Widgets mittels JavaScript oder TypeScript (GJS) zu definieren und zu steuern (vgl. \cite{astal}).
\begin{itemize}
    \item \textbf{Vorteile:} Dieses Tool erlaubt eine extrem schnelle und deklarative Entwicklung von Shell-Komponenten (wie Bars, Panels oder Widgets) und bietet eine sehr tiefe Integration in das GTK-Ökosystem.
    \item \textbf{Nachteile:} Obwohl Astal auf GTK basiert, erzwingt es die Nutzung von JavaScript/TypeScript als "Klebstoffsprache". Dies bricht die Homogenität der Codebasis, da der Backend-Code in Python vorliegt. Eine direkte Interaktion zwischen Python-Objekten und dem UI wäre erschwert. Zudem ist dieses Tool primär für Shell-Erweiterungen und weniger für eigenständige Anwendungsfenster konzipiert.
\end{itemize}

\subsubsection{Fazit} Die Entscheidung fiel auf GTK 3 mit Python, da es den besten Kompromiss aus nativer Integration, einheitlicher Sprachbasis (Python) und modernen Gestaltungsmöglichkeiten (CSS) bot.

\subsection{Architektur der GTK-Anwendung}
Die Architektur der Tapyre-Desktopanwendung ist darauf ausgelegt, eine schlanke, performante und nahtlos in die Desktop-Umgebung integrierte Benutzeroberfläche zu schaffen. Der zentrale Einstiegspunkt ist die Klasse \texttt{MyWindow}, die von \texttt{Gtk.Window} erbt. Diese Klasse ist für den gesamten Lebenszyklus des Fensters verantwortlich, von der Initialisierung der Widgets über die Konfiguration des Fensterverhaltens bis hin zur Behandlung von Benutzerinteraktionen.

\subsubsection{Fenster-Management mit GtkLayerShell}
Eine Kernanforderung an die Tapyre-Anwendung war es, nicht als herkömmliches Anwendungsfenster, sondern als eine Art Overlay oder Desktop-Widget zu fungieren, das bei Bedarf schnell aufgerufen werden kann. Um dieses Verhalten zu realisieren, wurde die Bibliothek \textbf{GtkLayerShell} eingesetzt. Diese Bibliothek ermöglicht es GTK-Anwendungen, das Layer-Shell-Protokoll des Wayland-Compositors zu nutzen, das für die Erstellung von Komponenten der Desktop-Umgebung (wie Panels, Hintergrundbilder oder eben Overlays) vorgesehen ist.

Die Konfiguration erfolgt durch gezielte Aufrufe nach der Initialisierung des Fensters:
\begin{itemize}
    \item \texttt{GtkLayerShell.init\_for\_window(self)}: Dieser Aufruf registriert das GTK-Fenster bei der Layer Shell und signalisiert dem Compositor, dass es als Teil der Desktop-Shell und nicht als normale Anwendung behandelt werden soll.
    \item \texttt{GtkLayerShell.set\_layer(self, GtkLayerShell.Layer.OVERLAY)}: Hiermit wird das Fenster auf der "Overlay"-Ebene platziert. Diese Ebene befindet sich über den normalen Anwendungsfenstern, aber unterhalb von systemeigenen Overlays wie z.B. dem Sperrbildschirm. Dies stellt sicher, dass Tapyre über anderen Anwendungen schwebt, wenn es sichtbar ist.
    \item \texttt{GtkLayerShell.set\_keyboard\_mode(self, GtkLayerShell.KeyboardMode.EXCLUSIVE)}: Dieser Modus gewährt dem Tapyre-Fenster den exklusiven Fokus auf die Tastatureingaben, solange es sichtbar ist. Dies ist entscheidend, damit Benutzereingaben zuverlässig erfasst werden und nicht versehentlich an das darunterliegende Fenster gesendet werden.
\end{itemize}
Durch den Einsatz von GtkLayerShell wird das Anwendungsfenster tief in die Desktop-Shell integriert und erhält den Charakter eines nativen System-Widgets, was maßgeblich zur angestrebten nahtlosen Benutzererfahrung beiträgt.

\subsubsection{Widget-Hierarchie und Layout}
Das Layout der Anwendung ist bewusst minimalistisch gehalten, um den Fokus des Benutzers auf die zentrale Eingabeaufforderung zu lenken. Die Struktur wird durch eine Hierarchie von GTK-Widgets realisiert, die ineinander verschachtelt sind.

\lstinputlisting[
style=python,
caption={Initialisierung der Widget-Hierarchie},
captionpos=b,
firstline=33,
lastline=49
]{sourcecode/tapyre/tapyre/src/frontend/gtk.py}

Die Hierarchie lässt sich wie folgt beschreiben:
\begin{enumerate}
    \item \textbf{\texttt{Gtk.Window}:} Das Hauptfenster (\texttt{MyWindow}) dient als Wurzelcontainer für alle anderen Elemente.
    \item \textbf{\texttt{Gtk.Box (vertikal)}:} Direkt im Fenster platziert ist eine vertikale Box (\texttt{self.box}), die als Hauptcontainer für das Layout dient. Sie hat die ID \texttt{\#main-box} für das CSS-Styling und sorgt dafür, dass die darin enthaltenen Elemente untereinander angeordnet werden.
    \item \textbf{\texttt{Gtk.Box (horizontal)}:} Innerhalb der vertikalen Box befindet sich eine horizontale Box (\texttt{input\_box}), die das Label und das Eingabefeld nebeneinander anordnet. Sie wird mit der Methode \texttt{pack\_start} in die vertikale Box gepackt, wobei die Parameter \texttt{False, False, 10} dafür sorgen, dass sie sich nicht ausdehnt und einen inneren Abstand von 10 Pixeln hat.
    \item \textbf{\texttt{Gtk.Label}:} Ein einfaches Label (\texttt{self.prompt\_label}) mit dem Text " > " dient als visueller Indikator für die Eingabeaufforderung, ähnlich einer Kommandozeile.
    \item \textbf{\texttt{Gtk.Entry}:} Das zentrale Element der Benutzeroberfläche ist das Eingabefeld (\texttt{self.entry}). Es erhält die ID \texttt{\#user-input} für das Styling. Durch den Aufruf \texttt{input\_box.pack\_start(self.entry, True, True, 0)} wird das Eingabefeld so konfiguriert, dass es den gesamten verfügbaren horizontalen Platz innerhalb der \texttt{input\_box} ausfüllt.
\end{enumerate}
Diese klare, hierarchische Anordnung von Containern und Widgets ist ein typisches Merkmal von GTK-Anwendungen und ermöglicht ein präzises und flexibles Layout-Management.

\subsubsection{Ereignisbehandlung und asynchrone Prozessausführung}
Die Interaktivität der Anwendung wird durch das Signal-System von GTK realisiert. Jede Benutzeraktion, wie eine Tastatureingabe oder das Drücken der Enter-Taste, löst ein Signal aus, mit dem eine Callback-Funktion verbunden werden kann.

\lstinputlisting[
style=python,
caption={Ereignisbehandlung und Threading},
captionpos=b,
firstline=51,
lastline=61
]{sourcecode/tapyre/tapyre/src/frontend/gtk.py}

Zwei zentrale Ereignisse werden in der Anwendung behandelt:
\begin{itemize}
    \item \textbf{Das \texttt{activate}-Signal:} Dieses Signal wird vom \texttt{Gtk.Entry}-Widget ausgelöst, wenn der Benutzer die Enter-Taste drückt. Es ist mit der Methode \texttt{on\_entry\_activate} verbunden. Diese Methode liest den Text aus dem Eingabefeld, leert es, und startet dann die Verarbeitung der Anfrage.
    \item \textbf{Das \texttt{key-press-event}:} Das Hauptfenster lauscht auf dieses Ereignis, um auf das Drücken der Escape-Taste zu reagieren und die Anwendung durch den Aufruf von \texttt{Gtk.main\_quit()} zu beenden.
\end{itemize}

Eine besondere Herausforderung bei der Anbindung an ein KI-System ist die potenziell lange Latenz der Agenten-Antworten. Würde die \texttt{agent.ask()}-Methode direkt im GTK-Main-Loop (dem Haupt-Thread der GUI) ausgeführt, würde die gesamte Benutzeroberfläche für die Dauer der Anfrage einfrieren.

Um dies zu verhindern, wird die Anfrage in einem separaten Thread ausgeführt. In \texttt{on\_entry\_activate} wird eine neue Instanz von \texttt{threading.Thread} erstellt, der die \texttt{self.agent.ask}-Methode als Ziel übergeben wird. Mit \texttt{thread.start()} beginnt die Ausführung im Hintergrund. Unmittelbar danach wird die GUI mit \texttt{self.hide()} ausgeblendet und der GTK-Main-Loop mit \texttt{Gtk.main\_quit()} beendet, um Ressourcen freizugeben. Der Agent kann seine Arbeit im Hintergrund-Thread beenden, ohne die Benutzererfahrung zu blockieren.

\subsection{Integration des CSS-Stylings}
Um das visuelle Design der Anwendung flexibel und unabhängig von der Anwendungslogik zu gestalten, nutzt Tapyre die integrierte CSS-Engine von GTK. Anstatt Farben, Abstände oder Schriftarten fest im Python-Code zu definieren, werden diese in einer externen CSS-Datei (\texttt{src/style/main.css}) deklariert.

\lstinputlisting[
language=css,
style=python,
caption={Ausschnitt aus der CSS-Datei für das Anwendungs-Styling},
captionpos=b
]{sourcecode/tapyre/tapyre/src/style/main.css}

Die Einbindung geschieht über einen \texttt{Gtk.CssProvider}, der die CSS-Datei lädt und für den gesamten Bildschirm (\texttt{Gdk.Screen}) bereitstellt. Im Code werden den GTK-Widgets über die Methode \texttt{set\_name()} eindeutige IDs zugewiesen (z.B. \texttt{\#main-box}, \texttt{\#user-input}). Diese IDs fungieren als CSS-Selektoren und erlauben es, die Widgets gezielt zu gestalten.

So wird beispielsweise über \texttt{\#main-box} der Hintergrund der Anwendung auf semi-transparentes Schwarz gesetzt und ein abgerundeter Rahmen definiert. Der Selektor \texttt{\#user-input} sorgt dafür, dass das Eingabefeld selbst keinen Hintergrund oder Rahmen hat und sich nahtlos in das Design einfügt. Dieser Ansatz entkoppelt das Design vollständig von der Logik und ermöglicht es, das Erscheinungsbild der Anwendung zu ändern, ohne eine einzige Zeile Python-Code anpassen zu müssen.

\section{Tapyre Paper Search}

Parallel zur Desktop-Anwendung wurde die \textbf{Tapyre Paper Search} Webanwendung entwickelt. Sie dient als leicht zugängliche, browserbasierte Schnittstelle für die im Backend implementierte semantische Such- und Embedding-Pipeline.

\subsection{Auswahl des Technologie-Stacks}

Für das Frontend der Paper-Search-Anwendung wurde ein moderner Web-Technologie-Stack gewählt, der auf Skalierbarkeit, eine schnelle Entwicklungs-Erfahrung und eine reaktive Benutzeroberfläche ausgelegt ist.

Als Framework wurde \textbf{Next.js} (vgl. \cite{nextjs}) gewählt, das auf der JavaScript-Bibliothek \textbf{React} (vgl. \cite{react}) aufbaut. Next.js bietet entscheidende Vorteile wie serverseitiges Rendern (SSR) und statische Seitengenerierung (SSG), die zu exzellenten Ladezeiten und einer besseren Suchmaschinenoptimierung führen. Die komponentenbasierte Architektur von React fördert zudem die Wiederverwendbarkeit von UI-Elementen und die Wartbarkeit des Codes.

Als Programmiersprache kommt \textbf{TypeScript} (vgl. \cite{typescript}) zum Einsatz. Als typisierte Obermenge von JavaScript erhöht es die Codequalität und Entwicklerproduktivität, indem es Fehler bereits während der Entwicklung durch statische Typüberprüfung erkennt und eine robustere Code-Basis sowie eine bessere Werkzeugunterstützung ermöglicht.

Für die Gestaltung der Benutzeroberfläche wurde auf \textbf{Tailwind CSS} (vgl. \cite{tailwindcss}) in Kombination mit der Komponentenbibliothek \textbf{shadcn/ui} (vgl. \cite{shadcnui}) gesetzt. Tailwind CSS ist ein Utility-First-Framework, das eine schnelle und flexible Gestaltung direkt im HTML-Markup erlaubt, ohne dass dedizierte CSS-Dateien pro Komponente notwendig sind. shadcn/ui ergänzt diesen Ansatz, indem es eine Sammlung barrierefreier und anpassbarer UI-Komponenten bereitstellt, die als hochwertige Grundlage für ein konsistentes und modernes Design dienen.

Um eine konsistente und isolierte Ausführungsumgebung zu gewährleisten und die Bereitstellung (Deployment) zu vereinfachen, ist die Anwendung für den Betrieb in einem \textbf{Docker-Container} konzipiert, wie durch das bereitgestellte \texttt{Dockerfile} ersichtlich ist.

\subsection{Evaluierung alternativer Frontend-Frameworks}

Vor der Festlegung auf Next.js und React wurden weitere führende Frontend-Frameworks evaluiert. Die Entscheidung basierte auf den spezifischen Anforderungen des Projekts, insbesondere der Notwendigkeit einer schnellen Prototypentwicklung, einer guten Entwicklererfahrung und der Verfügbarkeit eines robusten Ökosystems.

\subsubsection{Angular}
Angular ist ein umfassendes, von Google entwickeltes Framework (vgl. \cite{angular}), das auf TypeScript basiert und einem strikten Model-View-Controller (MVC)-Ansatz folgt.
\begin{itemize}
    \item \textbf{Vorteile:} Es bietet eine voll ausgestattete, meinungsstarke (opinionated) Architektur, die viele Entscheidungen vorgibt und sich gut für große Enterprise-Anwendungen eignet. Das integrierte Dependency-Injection-System und die umfangreichen Werkzeuge (Angular CLI) sind mächtig.
    \item \textbf{Nachteile:} Die steile Lernkurve und die hohe Komplexität wurden für ein schnelles Prototyping im Rahmen dieser Arbeit als hinderlich bewertet. Die strikte Struktur von Angular bietet weniger Flexibilität als React.
\end{itemize}

\subsubsection{Vue.js}
Vue.js (vgl. \cite{vuejs}) ist bekannt für seine einfache Erlernbarkeit und seine progressive Natur, die es erlaubt, es schrittweise in Projekte zu integrieren.
\begin{itemize}
    \item \textbf{Vorteile:} Die saubere Trennung von HTML-Templates, JavaScript-Logik und CSS-Styling (Single-File Components) ist sehr entwicklerfreundlich. Die Dokumentation gilt als eine der besten in der Branche.
    \item \textbf{Nachteile:} Obwohl das Ökosystem von Vue.js stark wächst, ist es immer noch kleiner als das von React. Für serverseitiges Rendering bietet Vue.js mit Nuxt.js eine Lösung, die jedoch zum Zeitpunkt der Entscheidung nicht die gleiche Reife und Verbreitung wie Next.js aufwies.
\end{itemize}

\subsubsection{Svelte}
Svelte (vgl. \cite{svelte}) verfolgt einen radikal anderen Ansatz: Anstatt die Arbeit zur Laufzeit im Browser durch ein virtuelles DOM zu erledigen, verschiebt Svelte diese Arbeit in einen Kompilierungsschritt.
\begin{itemize}
    \item \textbf{Vorteile:} Dies führt zu extrem performantem, Vanilla-JavaScript-Code ohne Framework-Overhead zur Laufzeit, was in sehr kleinen Bundle-Größen resultiert. Der Code ist oft kürzer und leichter verständlich.
    \item \textbf{Nachteile:} Svelte ist ein jüngeres Framework mit einem entsprechend kleineren Ökosystem und weniger verfügbaren Bibliotheken und Werkzeugen im Vergleich zu React.
\end{itemize}

\subsubsection{Fazit}
Die Entscheidung für \textbf{React} mit \textbf{Next.js} bot den besten Kompromiss aus Flexibilität, einem riesigen Ökosystem an Bibliotheken (einschließlich UI-Komponenten), exzellenter Community-Unterstützung und einer ausgereiften Lösung für serverseitiges Rendering. Dies ermöglichte eine schnelle und effiziente Entwicklung der Tapyre Paper Search Anwendung.

\subsection{Architektur der Frontend-Anwendung}
Die Webanwendung folgt einer komponentenorientierten Architektur, einem zentralen Paradigma von React. Diese Architektur fördert die Wiederverwendbarkeit und Kapselung von UI-Logik und -Struktur.

\subsubsection{Komponenten-Hierarchie und Komposition}
Die Anwendung ist hierarchisch aus Komponenten aufgebaut. An der Spitze steht die \texttt{app/page.tsx}, die die zentrale \texttt{PaperSearch}-Komponente rendert. Diese wiederum komponiert ihre Oberfläche aus kleineren, spezialisierten Bausteinen wie \texttt{InputGroup}, \texttt{Item} und \texttt{DropdownMenu}. Dieses Prinzip der Komposition erlaubt es, komplexe UIs aus einfachen, testbaren Teilen zusammenzusetzen.

\subsubsection{Zustandsverwaltung (State Management)}
Der Zustand der Anwendung wird lokal in der \texttt{PaperSearch}-Komponente mit dem \texttt{useState}-Hook von React verwaltet. Wichtige Zustandsvariablen sind \texttt{searchText} (Benutzereingabe), \texttt{searchResults} (API-Ergebnisse oder Fehler), und \texttt{isLoading} (zur Steuerung von Ladeindikatoren).

\subsection{Fehlerbehandlung und User Experience}
Eine robuste Anwendung muss auch Fehlerzustände und Ladezeiten adäquat behandeln, um eine positive User Experience (UX) zu gewährleisten.

\subsubsection{Asynchrones Feedback während der Suche}
Da die Suche im Backend dauern kann, wird dem Benutzer mit einem \texttt{Skeleton}-Loader visuelles Feedback gegeben. Während die \texttt{isLoading}-Variable auf \texttt{true} gesetzt ist, werden graue, pulsierende Platzhalter angezeigt, die die Struktur der zukünftigen Ergebnisse nachahmen. Dies verbessert die wahrgenommene Performance und reduziert die Unsicherheit des Benutzers.

\begin{figure}[h]
    \centering
    \includegraphics[width=1\textwidth]{figures/tapyre-skeleton.png}
    \caption{Skeleton Loader als visuelles Feedback während des Ladens}
    \label{fig:skeleton_loader}
\end{figure}

\subsubsection{Umgang mit API-Fehlern}
Netzwerk- oder Serverfehler werden im \texttt{catch}-Block der \texttt{handleSearch}-Funktion abgefangen. Anstatt die Anwendung abstürzen zu lassen, wird ein Fehlerobjekt im Zustand gespeichert. Die UI erkennt diesen Fehlerzustand und zeigt eine verständliche Meldung an.

\subsection{Styling-Architektur und Design-System}
Die visuelle Gestaltung basiert auf einem Utility-First-Ansatz mit Tailwind CSS.

\subsubsection{Utility-First mit Tailwind CSS}
Anstelle von traditionellem CSS werden Utility-Klassen (z.B. \texttt{flex}, \texttt{pt-4}) direkt im JSX-Markup kombiniert. Dies beschleunigt die Entwicklung, verhindert Namenskonflikte und sorgt für ein konsistentes Design. Nicht verwendete Klassen werden im Build-Prozess automatisch entfernt, was die finale CSS-Datei minimiert.

\subsubsection{Komponentenbasis mit shadcn/ui}
Aufbauend auf Tailwind CSS wird \texttt{shadcn/ui} verwendet. Anstatt eine Bibliothek zu installieren, wird der Code der benötigten Komponenten (z.B. Buttons, Dropdowns) direkt ins Projekt kopiert. Dies bietet maximale Kontrolle und Anpassbarkeit.

\subsection{Build- und Deployment-Prozess}
Für die Bereitstellung wird ein containerisierter Ansatz mit Docker und einer Multi-Stage-Build-Strategie verwendet, um ein schlankes und sicheres Produktions-Image zu erzeugen.

\lstinputlisting[
language=bash,
style=python,
caption={Multi-Stage-Build im Dockerfile der Webanwendung},
captionpos=b
]{sourcecode/website/Dockerfile}

\subsection{Benutzeroberfläche und Suche}
Die Benutzeroberfläche (siehe Abbildung \ref{fig:tapyre_paper_search_ui}) ist bewusst minimalistisch gehalten. Durch den Einsatz responsiver Designprinzipien passt sich die Darstellung zudem flexibel an unterschiedliche Bildschirmgrößen an und ist somit auch auf mobilen Endgeräten optimal nutzbar (vgl. Abbildung \ref{fig:tapyre_paper_search_mobile}). Das zentrale Element ist eine Eingabegruppe, die ein Textfeld für die Suchanfrage sowie Dropdown-Menüs zur Auswahl von Suchtyp und Ranking-Methode kombiniert.

\begin{figure}[h]
    \centering
    \includegraphics[width=0.9\textwidth]{figures/tapyre-paper-search.png}
    \caption{Tapyre Paper Search Benutzeroberfläche}
    \label{fig:tapyre_paper_search_ui}
\end{figure}

\begin{figure}[h]
    \centering
    \includegraphics[height=0.6\textwidth]{figures/tapyre-paper-search-mobile.png}
    \caption{Tapyre Paper Search auf einem mobilen Endgerät}
    \label{fig:tapyre_paper_search_mobile}
\end{figure}

Nach dem Absenden der Suche werden die Ergebnisse, wie in Abbildung \ref{fig:tapyre_paper_search_results} ersichtlich, unterhalb des Eingabebereichs als übersichtliche Liste dargestellt, die jeweils direkt auf die Publikation bei arXiv verlinkt.

\begin{figure}[h]
    \centering
    \includegraphics[width=0.9\textwidth]{figures/tapyr-paper-searc-results.png}
    \caption{Tapyre Paper Search mit Suchergebnissen}
    \label{fig:tapyre_paper_search_results}
\end{figure}


\appendix                       %% closes main document, appendix follows until end; only available in book-classes

\addpart*{Appendix}             %% adding Appendix to tableofcontents



\listoftables
\listoffigures
\lstlistoflistings
\nocite{*} %Es werden auch nicht referenzierte Literaturstellen aufgelistet
\bibliography{references}

\end{document}
